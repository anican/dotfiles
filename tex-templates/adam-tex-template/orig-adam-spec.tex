\documentclass[programming]{../../mfcs}
\course{CSE 143}{Computer Programming II}{Winter 2019}
\mfcsdate{Thursday, March 14, 2019 11:30pm}
\topic{HW8: Huffman Coding}
\hasmarginnotes
\usepackage{csquotes}
\usepackage{tabularx}
\newcommand{\dangerbendy}[1]{\marginnote{\vspace{-8.5em}\begin{center}\dbend\end{center}\scriptsize #1}}

\begin{document}

\vspace{-2.5em}
{\bf\color{colour} This assignment focuses on binary trees, priority queues, and recursion. This assignment is worth 30 points.
Turn in the following files using the link on the course website:}

\begin{itemize}\setlength{\itemindent}{-1.5em}
  \item \texttt{HuffmanCode.java} -- A class that represents a ``huffman code'' which
      is used to compress data. This file must contain a private static class named \texttt{HuffmanNode}.
  \item \texttt{secretmessage.short} -- A compressed hidden message that you will write
  \item \texttt{secretmessage.code} -- The corresponding huffman code for your hidden message
\end{itemize}

You will need the support file \texttt{HuffmanCompressor.java},
\texttt{BitInputStream.java}, and \texttt{BitOutputStream.java}; place these in the same
folder as your program or project.
The code you submit must work properly with the unmodified versions of the provided files.


\vspace{1em}

\begin{question}{ASCII and Binary Representation}
    We discovered on an earlier assignment that every \texttt{char} has an equivalent
    \texttt{int} value between 0 and 255.  (Recall,
    \verb+(int)'a' = 97+, \verb+(int)'z' = 122+, etc.)  In this assignment, we go a step
    further: computers actually see integers as \emph{binary} (1's and 0's).  Binary is a
    lot like the normal numbers we use, except it's based around 2 instead of 10.  For
    example, ``\texttt{1204}'' is written that way, because

    \vspace{-1em}

    $$1204 = \fbox{1} \cdot 10^3 + \fbox{2} \cdot 10^2 + \fbox{0} \cdot 10^1 + \fbox{5} \cdot 10^0$$

    \vspace*{-1em}

    Similarly, because

    \vspace*{-1em}

    $$97 = \fbox{0} \cdot 2^7 + \fbox{1} \cdot 2^6 + \fbox{1} \cdot 2^5 + \fbox{0} \cdot 2^4 + \fbox{0} \cdot 2^3 + \fbox{0} \cdot 2^2 +
    \fbox{0} \cdot 2^1 + \fbox{1} \cdot 2^0$$

    \vspace*{-0.2em}

    97 is ``\texttt{01100001}'' in binary!  In fact, all \emph{char}'s can be written using
    exactly 8 binary digits (bits).  This encoding, which is used pretty universally by
    computers, is called ASCII.
\end{question}

\vspace{-1em}

\begin{question}{Huffman Coding}
    \emph{Huffman coding} is an algorithm devised by David A. Huffman of MIT in 1952 for
    \emph{compressing} data to make a file occupy a smaller amount of space.  Unbelievably,
    this algorithm is still used today in a variety of very important areas.  For example,
    mp3s and jpgs both use Huffman Coding. The general idea behind Huffman Coding is the
    following:
    \begin{center}
        \textbf{What if we used fewer than 8 bits for characters that are more common?}
    \end{center}\vspace*{-0.7em}

    Consider the following (simple) example.  Imagine we have the following
    data:\vspace{-0.8em}

    {\begin{center}
    \fbox{\begin{minipage}{0.96\textwidth}
        \texttt{\underline{bbbb}aaaaaaaaaaaaaaaaaaaaaaaaaaaaaaaaaaaaaaaaaaaaaaaaaaaaaaaaaaaaaaaaaaaaaaaaaaaa}\\
        \texttt{\underline{ccc}aaaaaaaaaaaaaaaaaaaaaaaaaaaaaaaaaaaaaaaaaaaaaaaaaaaaaaaaaaaaaaaaaaaaaaaaaaaaa}\\
        \texttt{\underline{dd}aaaaaaaaaaaaaaaaaaaaaaaaaaaaaaaaaaaaaaaaaaaaaaaaaaaaaaaaaaaaaaaaaaaaaaaaaaaaaa}
    \end{minipage}}
    \end{center}}\vspace{-0.8em}

    \begin{minipage}{0.46\textwidth}
        \examheader{How many bits to store with ASCII Coding?}
        First, we find the ASCII code for each letter:\vspace*{-1.0em}
        \begin{center}
            \begin{tabular}{ccccccc}
                \texttt{a} & \to & \texttt{01100001}, &\hspace*{1.6em}& \texttt{b} & \to & \texttt{01100010},\\
                \texttt{c} & \to & \texttt{01100011}, &\hspace*{1.6em}& \texttt{d} & \to & \texttt{01100100}\\
            \end{tabular}
        \end{center}
        Since each line has 80 letters, and each letter code is 8 bits, the number of bits required is:
        \begin{center}\fbox{\textbf{$80 \cdot 8 \cdot 4 = 2560$}}\end{center}
        \vspace{0.7em}
    \end{minipage}\hfill
    \begin{minipage}{0.46\textwidth}
        \examheader{How many bits to store with Huffman Coding?}
        Since \verb+a+ is most frequent, we use a short code for it, then we use the next longest code for \verb+b+, etc:\vspace{-0.5em}
        \begin{center}
            \begin{tabular}{ccccccc}
                \texttt{a} & \to & \texttt{1}, &\hspace*{1.6em}& \texttt{b} & \to & \texttt{00},\\
                \texttt{c} & \to & \texttt{011}, &\hspace*{1.6em}& \texttt{d} & \to & \texttt{010}\\
            \end{tabular}
        \end{center}\vspace*{-0.5em}
        This data has 229 \verb+a+'s, 4 \verb+b+'s, 3 \verb+c+'s, and 2 \verb+d+'s.  Since we need one bit for \verb+a+,
        two for \verb+b+, and three for \verb+c+ and \verb+d+, the total count of bits is:
        \begin{center}\fbox{\textbf{$229 \cdot 1 + 4 \cdot 2 + 3 \cdot 3 + 2 \cdot 3 = 255$}}\end{center}
        \end{minipage}

        \vspace{1em}

        \centering Woah! By changing the coding, we compressed the data by a factor of 10!!

\vspace{0.5em}

\justifying\noindent
In this assignment, you will create classes \texttt{HuffmanCode} and \texttt{HuffmanNode}
to be used in compression of data You are provided with a client
\texttt{HuffmanCompressor.java} that handles user interaction and calls your huffman code
methods to compress and decompress a given file.
\end{question}

\newcommand{\qquote}[1]{\textquoteleft\texttt{#1}\textquoteright}

\begin{question}{Making a Huffman Code}
    In the previous example, we magically arrived at the special binary sequences to be used
    as codes.  In this section, we explain the algorithm to create these special binary
    sequences.

    Throughout this section, we will use the following example:\vspace*{1.0em}

\renewcommand{\FancyVerbFormatLine}[1]{#1}
\begin{Verbatim}[fontsize=\small,frame=single,label=simple-spec-example.txt]
aba ab cabbb
\end{Verbatim}


    \subquestion*{Step 1: Count the Frequencies of the Characters}
    The file \texttt{simple-spec-example.txt} has the characters \verb+'a'+, \verb+'b'+, \verb+'c'+, and \verb+' '+.  Counting up the frequencies,
    we get the following:
    \begin{center}
    \begin{tabular}{|c|c|}\hline
    \textbf{Character} & \textbf{Count}\\\hline
    \qquote{ }       & 2\\\hline
    \qquote{a}       & 4\\\hline
    \qquote{b}       & 5\\\hline
    \qquote{c}       & 1\\\hline
    \end{tabular}
    \end{center}

    \subquestion*{Step 2: Create a ``Priority Queue'' Ordered By Frequency}
    Since our ultimate goal is to create a code based on frequencies, we need to use a data
    structure that helps us keep track of the order of the letters based on frequencies.  We
    will use a \emph{priority queue} for this.  A \emph{priority queue} is a queue that is
    ordered by \emph{priorities} (in this case frequencies) instead of FIFO order.  In other
    words, we can ask the priority queue to insert a new element (\texttt{add(element)}) and
    we can ask it for the highest priority (\texttt{remove()}).
    \dangerbendy{In Java, if you print out a priority queue, it \emph{will not}
        appear in the order you expect it to.  This is EXPECTED behavior.}

    \vspace{0.5em}

    For huffman coding, we want to remove the node with \emph{lowest} frequency first.
    (Intuitively, the reason is that the things we remove first will end up having the
    longest codes).

    \vspace{0.5em}
    \newcommand{\huffmannode}[3][\qquote]{\small \begin{minipage}{3em}\begin{center}#1{\texttt{#2}}\\\vspace{0.2em}\hrule\vspace{0.2em}\emph{freq: }#3\end{center}\end{minipage}}

    \newcommand{\huffmantree}[2][4em]{
        \begin{minipage}{#1}
            \vspace*{0.5em}
            \binarytree{#2}

            \vspace*{0.5em}
        \end{minipage}
    }

    We begin by creating a node for each letter in our text:

    \begin{center}
    \scalebox{0.6}{\begin{tabular}{c|c|c|c|c|c}\cline{2-5}
        pq $\longleftarrow$ &
        \huffmantree{"\huffmannode{c}{1}"}&
        \huffmantree{"\huffmannode{ }{2}"}&
        \huffmantree{"\huffmannode{a}{4}"}&
        \huffmantree{"\huffmannode{b}{5}"}&
        $\longleftarrow$ \\\cline{2-5}
    \end{tabular}}
    \end{center}

    \subquestion*{Step 3: Combine the Nodes Until Only One is Left}
    Now that we have a priority queue of the nodes, we want to put them together into a
    tree.  To do this, we repeatedly remove the smallest two (the ones at the front of the
    priority queue) and create a new node with them as children. Note that the priority
    queue is \emph{arbitrary} in how it breaks ties.

    \vspace{0.5em}

    \begin{minipage}{0.29\textwidth}
    \examheader{(3a) Remove Two Smallest}
    First, remove the smallest two nodes from the priority queue:\vspace{-1em}
    \begin{center}
    \scalebox{0.8}{
    \hfill \huffmantree{"\huffmannode{c}{1}"} \hfill and \hfill \huffmantree{"\huffmannode{ }{2}"}\hfill
    }

    \vspace{0.5em}

    \scalebox{0.6}{
    \hspace*{-1.5em}\begin{tabular}{c|c|c|c}\cline{2-3}
        pq $\longleftarrow$ &
        \huffmantree{"\huffmannode{a}{4}"}&
        \huffmantree{"\huffmannode{b}{5}"}&
        $\longleftarrow$ \\\cline{2-3}
    \end{tabular}}
    \end{center}
    \vspace{0.49em}
    \end{minipage}
    \hfill
    \begin{minipage}{0.30\textwidth}
    \examheader{(3b) Combine Them Together}
    {\small Then, create a new node. The frequency is the sum of two nodes:}\vspace{-1em}
    \begin{center}
    \hspace*{-2em}\scalebox{0.6}{
    \huffmantree{"\huffmannode[]{\;}{3}" -> {"\huffmannode{c}{1}", "\huffmannode{ }{2}"}}
    }

    \hspace*{0em}\scalebox{0.6}{
    \begin{tabular}{c|c|c|c}\cline{2-3}
        pq $\longleftarrow$ &
        \huffmantree{"\huffmannode{a}{4}"}&
        \huffmantree{"\huffmannode{b}{5}"}&
        $\longleftarrow$ \\\cline{2-3}
    \end{tabular}}
    \end{center}
    \end{minipage}
    \hfill
    \begin{minipage}{0.34\textwidth}
    \examheader{(3c) Add Back To Priority Queue}
    Now that we have a ``new node'', add it back to the priority queue:
    \begin{center}
    \scalebox{0.52}{
    \begin{tabular}{c|c|c|c|c}\cline{2-4}
        pq $\longleftarrow$ &
        \huffmantree[10em]{"\huffmannode[]{\;}{3}" -> {"\huffmannode{c}{1}", "\huffmannode{ }{2}"}}&
        \huffmantree{"\huffmannode{a}{4}"}&
        \huffmantree{"\huffmannode{b}{5}"}&
        $\longleftarrow$ \\\cline{2-4}
    \end{tabular}}
    \end{center}
    \vspace{0.9em}
    \end{minipage}

    \vspace{1.0em}

    \centering We repeat this process until there is only one node left in the priority
    queue.

    \justifying\noindent
    Here are the remaining steps (each time, we remove the two smallest frequencies, combine
    them, and put the result back):

    \hspace*{-2.5em}\scalebox{0.52}{
    \begin{tabular}{c|c|c|c|c}\cline{2-4}
        pq $\longleftarrow$ &
        \huffmantree[10em]{"\huffmannode[]{\;}{3}" -> {"\huffmannode{c}{1}", "\huffmannode{ }{2}"}}&
        \huffmantree{"\huffmannode{a}{4}"}&
        \huffmantree{"\huffmannode{b}{5}"}&
        $\longleftarrow$ \\\cline{2-4}
    \end{tabular}}
    {\scriptsize then,}
    \scalebox{0.52}{
    \begin{tabular}{c|c|c|c}\cline{2-3}
        pq $\longleftarrow$ &
        \huffmantree{"\huffmannode{b}{5}"}&
        \huffmantree[13em]{"\huffmannode[]{\;}{7}" -> {
        {"\huffmannode[]{\;}{3}" -> {"\huffmannode{c}{1}", "\huffmannode{ }{2}"}},
        "\huffmannode{a}{4}"}}&
        $\longleftarrow$ \\\cline{2-3}
    \end{tabular}}
    {\scriptsize then,}
    \scalebox{0.52}{
    \begin{tabular}{c|c|c}\cline{2-2}
        pq $\longleftarrow$ &
        \huffmantree[14em]{
            "\huffmannode[]{\;}{12}" -> {
        "\huffmannode{b}{5}",
        {"\huffmannode[]{\;}{7}" -> {
        {"\huffmannode[]{\;}{3}" -> {"\huffmannode{c}{1}", "\huffmannode{ }{2}"}},
        "\huffmannode{a}{4}"}}}
        }&
        $\longleftarrow$ \\\cline{2-2}
    \end{tabular}}

    \vspace{1em}

    \noindent Now that we only have one node left, we can use the tree we constructed to
    create the huffman codes!

    \subquestion*{Step 4: Read Off The Huffman Codes}
    \noindent At this point, the frequencies of the letters have already been taken into
    account; so, we no longer even look at them.  Thus, the tree we've constructed looks
    like the following:

\vspace{1em}

\hspace*{5em}
\begin{minipage}{0.3\textwidth}
    \scalebox{0.8}{\binarytree[font=\normalsize][4]{
        r1/"" ->[edge label=0, swap] "\qquote{b}",
        r1/"" ->[edge label=1] {
            r2/"" ->[edge label=0, swap] {
                r3/"" ->[edge label=0, swap] "\qquote{c}",
                r3/"" ->[edge label=1] "\qquote{ }"
            },
            r2/"" ->[edge label=1] "\qquote{a}"
        }
    }}
\end{minipage}
\hfill
\begin{minipage}{0.3\textwidth}
\renewcommand{\FancyVerbFormatLine}[1]{#1}
\begin{Verbatim}[fontsize=\small,frame=single,label=simple-spec-example.code]
98
0
99
100
32
101
97
11
\end{Verbatim}
\end{minipage}
\hspace*{5em}

\vspace{1em}

    \noindent To figure out the huffman code for a letter, we traverse the tree from the
    root to the node with the letter in it.  When we go left,  we print a 0 and if we go
    right, we print a 1.  For example, \qquote{c} would be \texttt{100}, because we
    go right, then left, then left to reach it.

    \vspace{0.5em}

    \noindent Just like in 20 questions, we will output the tree in ``standard format''.
    Notice that the only actual information is in the leaves of the tree.  So, the
    \texttt{code} file, which you can get by asking the main to ``make a code'', will
    consist of line pairs: the first line will be the ASCII value of the character in the
    leaf and the second line will be the huffman code for that character.  For example,
    the output of the tree we just constructed would look like the above.  The leaves should
    appear in traversal order.
\end{question}

\begin{question}{Huffman Compression and Decompression}
Now that we know how to construct a huffman code, we are ready to understand the huffman
compression and decompression algorithms.  Here is an overview of how they work:

\tikzset{proc/.style={
    rectangle,
    rounded corners,
    draw=black,
    very thick,
    align=center,
    minimum height=1.5em,
    minimum width=8em,
    text width=7em
}}

\vspace{0.1em}

\begin{minipage}{0.3\textwidth}
\scalebox{0.62}{
\binarytree[font=\normalsize][5]{
        inp/"Input File (\texttt{file.txt})"[proc] ->[edge label=You do this, swap]
            code/"Huffman Code (\texttt{file.code})"[proc],
        code -> comp/"Compressed File (\texttt{file.short})"[proc, below left=5em and 5em of code],
        inp ->[edge label=\begin{minipage}{0.9\textwidth}\texttt{HuffmanCompressor}\\handles this\end{minipage}]
            comp/"Compressed File (\texttt{file.short})"[proc],
        comp ->[edge label=You do this] "Decompressed File (\texttt{file.new})"[proc, below left=8em and 5em of comp],
        code -> "Decompressed File (\texttt{file.new})"[proc]
}}
\end{minipage}
\hfill
\begin{minipage}{0.65\textwidth}
\examheader{Compression Algorithm}\vspace{0.5em}
{\small \begin{enumerate}[(1)]\setlength{\itemindent}{-1em}
    \item \texttt{HuffmanCompressor} generates the frequencies in a given file.
    \item \texttt{HuffmanCode} creates the huffman code from the frequencies.
    \item \texttt{HuffmanCode} writes out the huffman codes to a \texttt{.code} file.
    \item \texttt{HuffmanCompressor} writes a \texttt{.short} file.
\end{enumerate}}

\examheader{Decompression Algorithm}\vspace{0.5em}
{\small \begin{enumerate}[(1)]\setlength{\itemindent}{-1em}
    \item \texttt{HuffmanCode} reads in a \texttt{.code} file in standard format.
    \item \texttt{HuffmanCode} writes out a \texttt{.new} file.
\end{enumerate}}

\end{minipage}



\end{question}
\newpage

\begin{question}{\texttt{HuffmanNode}}
    The contents of the \texttt{HuffmanNode} class are up to you. Though we have studied
    trees of \texttt{int}s, you \emph{should} create nodes specific to solving this problem.
    Your \texttt{HuffmanNode} class must be a private static inner class within \texttt{HuffmanCode},
    and should have at least one constructor used by your tree.
    Your node's fields \emph{must} be public. \texttt{HuffmanNode} should not contain \emph{any actual huffman coding logic}.
    It should only represent a single node of the tree.
\end{question}

\begin{question}{\texttt{HuffmanCode}}
    This class represents a huffman code for a particular message. It keeps track of a
    binary tree constructed using the huffman algorithm.
\vspace{1em}

\subquestion*{\texttt{HuffmanCode} should have the following constructors:}
\begin{method}{public}{HuffmanCode}{int[] frequencies}
    This constructor should initialize a new \texttt{HuffmanCode} object using the algorithm
    described for making a code from an array of frequencies.  \texttt{frequencies} is an
    array of frequencies where \texttt{frequences[i]} is the count of the character with
    ASCII value \texttt{i}. Make sure to use a \texttt{PriorityQueue} to build the huffman
    code.
\end{method}

\begin{method}{public}{HuffmanCode}{Scanner input}
    This constructor should initialize a new \texttt{HuffmanCode} object by reading in a
    previously constructed code from a \texttt{.code} file. You may assume the
    \texttt{Scanner} is not \texttt{null} and is always contains data encoded in legal,
    valid standard format.
\end{method}

\subquestion*{\texttt{HuffmanCode} should also implement the following methods:}
\begin{method}{public void}{save}{PrintStream output}
    This method should store the current huffman codes to the given output stream in the
    standard format described above.
\end{method}

\begin{method}{public void}{translate}{BitInputStream input, PrintStream output}
    This method should read individual bits from the input stream and write the
    corresponding characters to the output. It should stop reading when the
    \texttt{BitInputStream} is empty. You may assume that the input stream contains a legal
    encoding of characters for this tree's huffman code.  \textbf{See below for the methods
    that a \texttt{BitInputStream} has.}
\end{method}
\end{question}

\vspace{-0.5em}

\begin{question}{\texttt{BitInputStream}}
The provided \texttt{BitInputStream} class reads data bit by bit.  This will be useful for
the \texttt{translate} method in \texttt{HuffmanCode}.  \texttt{BitInputStream} has the
following methods:

\begin{method}{public int}{nextBit}{}
    This method returns the next bit in the input stream.  If there is no such bit, then it
    throws a \texttt{NoSuchElementException}.
\end{method}

\begin{method}{public boolean}{hasNextBit}{}
    This method returns true if the input stream has at least one more bit and false
    otherwise.
\end{method}

The interface for \texttt{BitInputStream} looks very much like a \texttt{Scanner}, and it
should be used similiarly.
\end{question}

\begin{question}{Implementation Details}
Your program should exactly reproduce the format and general behavior demonstrated on the
output comparison tool. Note that this assignment has two mostly separate parts: creating a
huffman code and decompressing a message using your huffman code.  Of the four methods to
implement, two are relevant to each part.

\vspace{0.5em}

\subquestion*{Creating A Huffman Code}
You will write methods to (1) create a huffman code from an array of frequencies and (2)
write out the code you've created in standard format.

\subsubquestion*{Frequency Array Constructor}
You should use the algorithm described in the ``Making a Huffman Code'' section to implement
this constructor.  You will need to use \texttt{PriorityQueue<E>} in the \texttt{java.util}
package.

\vspace{0.5em}

The only difference between a priority queue and a standard queue is that it uses the
natural ordering of the objects to decide which object to dequeue first, with objects
considered ``less'' returned first.  You will be putting subtrees into your priority queue,
which means you'll be adding values of type \texttt{HuffmanNode}.

\vspace{0.5em}

This means that your \texttt{HuffmanNode} class will have to implement the
\texttt{Comparable<E>} interface.  It should use the frequency of the subtree to determine
its ordering relative to other subtrees, with lower frequencies considered ``less'' than
higher frequencies.  If two frequencies are equal, the nodes are too.

\vspace{0.5em}

Remember that, in order to make our code more flexible we should be declaring variables with
their interface types when possible. This means you should define your \texttt{PriorityQueue} variables with
the \texttt{Queue} interface.

\vspace{0.5em}

The Huffman solution is not unique.  You can obtain any one of several different equivalent
trees depending upon how certain decisions are made.  If you implement it as we have
specified, then you should get exactly the same tree for any particular implementation
of \texttt{PriorityQueue}.  Make sure that you use the built-in \texttt{PriorityQueue} class
and that when you are combining pairs of values taken from the priority queue, you make the
first value removed from the queue the left subtree and you make the second value removed
the right subtree.

\subquestion*{Decompressing A Message}
You will write methods to (1) read in a \texttt{.code} file created with \texttt{save()} and
(2) translate a compressed file back into a decompressed file.

\subsubquestion*{{\tt Scanner} Constructor}
This constructor will be given a \texttt{Scanner} that contains data produced by
\texttt{save()}.  In other words, the input for this constructor is the output you produced
into a \texttt{.code} file.  The goal of this constructor is to re-create the huffman tree
from your output.  Note that the frequencies are irrelevant for this part, because the tree
has already been constructed; so, you should set all the frequencies to some standard value
(such as 0 or -1) when creating \texttt{HuffmanNode}s in this constructor.
\dangerbendy{See lecture slides for example of this algorithm}

\vspace{0.5em}

Remember that the standard code file format is a series of pairs of lines where the first
line has an integer representing the character's ASCII value and the second line has the
code to use for that character. You might be tempted to call \texttt{nextInt()} to read the
integer and \texttt{nextLine()} to read the code, but remember that mixing token-based
reading and line-based reading is not simple. Here is an alternative that uses a method
called \texttt{parseInt} in the \texttt{Integer} class that allows you to use two successive
calls on \texttt{nextLine()}:
\begin{lstlisting}
int asciiValue = Integer.parseInt(input.nextLine());
String code = input.nextLine();
\end{lstlisting}


\subsubquestion*{\tt translate}
This method takes in a \texttt{BitInputStream} representing a previously compressed message
and outputs the original decompressed message. \texttt{BitInputStream} can be used in a very
similar way to a \texttt{Scanner}; see the description of its methods on page 4.

This method reads sequences of bits that represent encoded characters to figure out what the
original characters must have been.  The algorithm works as follows:
\begin{itemize}
    \item Begin at the top of the tree
    \item For each bit read in from the \texttt{BitInputStream}, if it's a 0, go left, and
        if it's a 1, go right.
    \item Eventually, we will hit a leaf node.  Once we do, write out the integer code for
        that character to the output using the following \texttt{PrintStream} method:
\begin{center}
\verb+public void write(int b)+
\end{center}
    \item Then, go back to the top of the tree, and do the process over again.
        \dangerbendy{Do not use \texttt{print} instead of \texttt{write}!}
\end{itemize}

You will have to be careful if you use recursion in your decode method.  Java has a limit on
the stack depth you can use.  For a large message, there are hundreds of thousands of
characters to decode.  It would not be appropriate to write code that requires a stack that
is hundreds of thousands of levels deep.  You might be forced to write some or all of this
using loops to make sure that you don't exceed the stack depth.
\end{question}

\begin{question}{Translate Example}
Suppose we have the following \texttt{.code} and \texttt{.short} files:\\

\hspace*{3em}
\begin{minipage}{0.4\textwidth}
\scalebox{0.8}{\binarytree[font=\normalsize][4]{
    r1/"" ->[edge label=0, swap] "\qquote{b}",
    r1/"" ->[edge label=1] {
        r2/"" ->[edge label=0, swap] {
            r3/"" ->[edge label=0, swap] "\qquote{c}",
            r3/"" ->[edge label=1] "\qquote{ }"
        },
        r2/"" ->[edge label=1] "\qquote{a}"
    }
}}
\end{minipage}
\hfill
\begin{minipage}{0.5\textwidth}
\renewcommand{\FancyVerbFormatLine}[1]{#1}
\begin{Verbatim}[fontsize=\small,frame=single,label=simple-spec-example.short]
1101110111010110011000
\end{Verbatim}
\end{minipage}\hspace*{3em}

\vspace{1em}

Read \texttt{1}, go right. Read \texttt{1}, go right. \qquote{a} is a leaf.  Output \qquote{a}.
\hfill (Input: \texttt{\underline{11}0101110101100})

Read \texttt{0}, go left.  \qquote{b} is a leaf.  Output \qquote{b}.
\hfill (Input: \texttt{11\underline{0}101110101100})

Read \texttt{1}, go right.  Read \texttt{0}, go left.  Read \texttt{1}, go right.  \qquote{ } is a leaf.  Output \qquote{ }.
\hfill (Input: \texttt{110\underline{101}110101100})

Read \texttt{1}, go right. Read \texttt{1}, go right. \qquote{a} is a leaf.  Output \qquote{a}.
\hfill (Input: \texttt{110101\underline{11}0101100})

Read \texttt{0}, go left.  \qquote{b} is a leaf.  Output \qquote{b}.
\hfill (Input: \texttt{11010111\underline{0}101100})

Read \texttt{1}, go right.  Read \texttt{0}, go left.  Read \texttt{1}, go right.  \qquote{ } is a leaf.  Output \qquote{ }.
\hfill (Input: \texttt{110101110\underline{101}100})

Read \texttt{1}, go right.  Read \texttt{0}, go left.  Read \texttt{0}, go right.  \qquote{c} is a leaf.  Output \qquote{c}.
\hfill (Input: \texttt{110101110101\underline{100}})

\vspace{1em}

So, the decompressed text is ``\texttt{ab ab c}''.

\end{question}

\begin{question}{Creative Aspect (\texttt{secretmessage.short} and \texttt{secretmessage.code})}
Along with your program you should turn in files named \texttt{secretmessage.short} and
\texttt{secretmessage.code} that represent a ``secret'' compressed message from you to your
TA and its code file. The message can be anything you want, as long as it is not offensive.
Your TA will decompress your message with your tree and read it while grading.
\end{question}


\newpage

\begin{question}{Style Guidelines and Grading}
Part of your grade will come from appropriately utilizing binary trees and recursion to
implement HuffmanCode as described previously.  We will also grade on the elegance of your
recursive algorithm; don't create special cases in your recursive code if they are
unnecessary.

\vspace{0.5em}

\subquestion*{\texttt{x = change(x)}}
An important concept introduced in lecture was called \texttt{x = change(x)}. This idea is
related to proper design of recursive methods that manipulate the structure of a binary
tree in order to reduce redundancy for cases dealing with left/right. You should use the
\texttt{x = change(x)} pattern where appropriate to remove such redundancy.

\subquestion*{Avoid Redundancy}
Create ``helper'' method(s) to capture repeated code.  As long as all extra methods you
create are \texttt{private} (so outside code cannot call them), you can  have additional
methods in your class beyond those specified here.  If you find that multiple methods in
your class do similar things, you should create helper method(s) to capture the common code.

\subquestion*{Data Fields}
Properly encapsulate your objects by making your fields \texttt{private}. (You can and
\emph{should} make fields in \texttt{HuffmanNode} public.) Avoid unnecessary fields;
use fields to store important data of your object but not temporary data used in only one
place. Always initialize fields in a constructor or method, never at declaration.

\subquestion*{Java Style Guidelines}
Appropriately use control structures like loops and if/else statements.  Avoid redundancy
using techniques such as methods, loops, and factoring common code out of if/else
statements. Properly use indentation, good variable names, and types. Do not have any lines
of code longer than 100 characters. Please refer to the \href{https://courses.cs.washington.edu/courses/cse143/19wi/homework/cse143-style-guide2/javaguide.html}{Style Guide}
and the \href{https://courses.cs.washington.edu/courses/cse143/19wi/style.shtml}{General Style Deductions} for a more complete list.

\end{question}
\end{document}
