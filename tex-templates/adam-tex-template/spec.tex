\documentclass[programming]{../../mfcs}
\course{CSE 431}{Complexity Theory}{Autumn 2019}
\mfcsdate{Monday, October 28, 2019}
\topic{Homework 4}
\hasmarginnotes
\usepackage{csquotes}
\usepackage{tabularx}
\newcommand{\dangerbendy}[1]{\marginnote{\vspace{-8.5em}\begin{center}\dbend\end{center}\scriptsize #1}}

\begin{document}
Anirudh Canumalla
\vspace{-2.5em}
% {\bf\color{colour} This assignment focuses on Turing machines.}

\begin{question}{1. (20 points)}
    \textbf{Definition: } A language $B$ is recursively-enumerable complete iff $B$ is
    Turing-recognizable and for all Turing-recognizable languages $A, A \leq_m B $.
\end{question}


\end{document}
